\section{Conclusion}
\label{conclusion}
%Existing prefix KV reuse systems do not always reduce TTFT, especially when disk
%I/O latency is involved in large-scale LLM services. We propose \pname{}, a
%multi-tier prefix KV storage system to 
%minimize I/O delay by only loading important KVs. Simply applying existing important token identification
%algorithms is suboptimal, as the reduction in I/O is limited. Therefore, we
%first introduce the I/O-efficient \techa{} algorithm to identify important KVs
%with minimal I/O. Then, we propose \techb{} to optimize storage and caching,
%further reducing TTFT. Our experiments show that \pname{} reduces TTFT by up to
%2.8$\times$ compared to state-of-the-art systems, while maintaining comparable
%inference accuracy.
\zrdnew{Existing prefix KV storage systems do not always reduce TTFT, especially when disk
	I/O latency becomes a bottleneck in large-scale LLM services. 
	We propose \pname{}, an importance-informed three-tiered prefix KV caching and prefetching system 
	that minimizes I/O delay by selectively loading and prefetching important KVs. 
	First, we introduce the I/O-efficient \techa{} algorithm to identify important KVs with minimum data movement. 
	Then, we propose \technew{}, which employs cross-layer prefetching to overlap I/O with computation 
	and further reduce disk traffic. 
	Finally, we propose \techb{} to optimize storage and caching,
	further reducing TTFT.
	Our experiments show that \pname{} reduces TTFT by up to 1.75$\times$ compared to state-of-the-art systems 
	while maintaining comparable inference accuracy.
}
\section*{Acknowledgments}
\fv{
We sincerely thank the anonymous reviewers for their constructive suggestions. 
This work was supported in part by the National Key Research and Development Program of China (2023YFB4502100), 
the National Science Foundation of China (62172361), 
the Major Projects of Zhejiang Province (LD24F020012), 
the Open Project Program of Wuhan National Laboratory for Optoelectronics (2023WNLOKF005),  
and the Pioneer and Leading Goose R\&D Program of Zhejiang Province (2024SSYS0002).
}
